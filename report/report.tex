\documentclass[titlepage]{article}

\usepackage[utf8]{inputenc}
\usepackage{graphicx}
\usepackage{listings}
\usepackage[bookmarks=true]{hyperref}
\usepackage[all]{hypcap}        % needed to help hyperlinks direct correctly;

\title{SongBook\\ \large Application programming – Java and XML technologies}
\author{Michał Kowalski\\Maciej Borkowski}
\date{}

\begin{document}
\maketitle

\section{Introduction}
SongBook is, without a surprise, a song book. It allows users to see lyrics and
chords for songs, grouped by artists/authors, stored in application's directory
in XML format. Adding/deleting/modifying of songs/artists can be done by
adding/deleting/modyfing proper XML files. Application is written in Java, using
JAXB and Swing.

\section{Data model}
All application's data is stored in 3 types of XML files:
\begin{itemize}
  \item index files,
  \item artist files,
  \item song files.
\end{itemize}
All files, except index ones, are loaded on-demand (lazy loading), which reduces
application's start time and required memory, when database grows, and allows
for changing data at runtime. Those files are parsed using JAXB into instances
of classes.

\subsection{Index file}
Index file, as shown on fig. \ref{lst:index}, contains a list of artist entries.
Each entry contains a name of artist and its ID. To find a data for artists,
file name must be equal to artist's ID. Index file and artist entry have their
appropiate Java classes for binding purposes.

\begin{figure}
\lstinputlisting[breaklines=true]{../test_db/index1.xml}
\caption{Example file containing list of authors}
\label{lst:index}
\end{figure}

\subsection{Author file}
Author/artist file, as shown on fig. \ref{lst:artist}, contains a list of song
entries and a description of author. Song entry contains song name and ID, and
latter of those two is used for finding file with data for each song. Author
file and song entry have their appropiate Java classes for binding purposes.

\begin{figure}
\lstinputlisting[breaklines=true]{../test_db/artists/jenny_donelly.xml}
\caption{Example file for author containing list of his songs}
\label{lst:artist}
\end{figure}

\subsection{Song file}
Song file, shown on fig. \ref{lst:song}, is lowest file in data hierarchy. This
file contains data for song - its lyrics and chords, represented in Java as
simple list of strings, so only file must have own class for binding.

\begin{figure}
\lstinputlisting[breaklines=true]{../test_db/songs/song_a.xml}
\caption{Example file for song}
\label{lst:song}
\end{figure}

\section{GUI}

\end{document}